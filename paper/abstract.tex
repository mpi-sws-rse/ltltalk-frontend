We present \tool, a
% community-powered
natural language interface for describing
high level task specifications for robots
that are compiled into robot actions.
\tool starts with a formal core language for task planning that allows
expressing rich temporal specifications and
uses a semantic parser to provide a natural language interface.
\tool provides immediate visual feedback by executing an automatically
constructed plan of the task in a graphical user interface.
This allows the user to resolve potentially ambiguous interpretations.
\tool extends itself via \emph{naturalization}: users of \tool can
define new commands, which are generalized and added as new rules to the core language,
gradually growing a more and more natural task specification language.
%
Unlike other task-specification systems, \tool enables natural language
interactions while maintaining the expressive power and formal precision of a programming language.
We show through an initial user study that natural language interactions and generalization
can considerably ease the description of tasks.
Moreover, over time, users employ more and more concepts outside of the initial core language.
Such extensions are available to the \tool community, and users can use concepts that others have defined.
