We present \tool, a natural language interface to describing 
high level task specifications for robots which can be automatically
compiled into correct-by-construction reactive plans.
Internally, \tool defines a core language for task planning that includes
predicates on the world, fluents (robot actions) that modify the state of the world, 
as well as temporal constraints.
A planner takes a program in the core language and automatically generates a (possibly reactive)
plan.
Externally, \tool provides a natural language interface to users and a visual feedback
mechanism.
It uses semantic parsing techniques from natural language to compile user utterances into
programs in the core language.
Then, it provides immediate visual feedback by executing the automatically constructed plan in the world
and allows the user to select between potentially ambiguous interpretations.
Unlike other systems, \tool enables natural language interactions while maintaining the expressive
power of a programming language. 
Additionally, the system learns through repeated interactions and generalizes learnt concepts.
We show through an initial user study that natural language interactions and generalization
can considerably ease the description of tasks, and moreover, over time,
users employ more and more defined concepts, not necesarily defiend by themselves. 
