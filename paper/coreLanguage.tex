\section{Core Language}


The user of our system instructs a single robot which moves in a world
consisting of several connected rooms. These rooms contain a number of items,
which can have different properties; for simplicity we consider here different
colors and shapes. The robot can modify this world by moving to a location
containing an item, by picking up and dropping items, possibly at different
locations.

\subsection{Syntax}
While the abstract world in which our robot operates is relatively simple,
the core language used to instruct the robot is sufficiently expressive to
allow for interesting scenarios. \autoref{fig:core-syntax} illustrates the
syntax of the core language.
\todo[inline]{(Eva) I have based this on syntaxsemantics.md. That document does not
have the select statement though. Do we have this?}
\begin{figure}
  Control flow:
  \begin{lstlisting}
  Stmt -> Act | Stmt; Stmt
  Stmt -> repeat n times Stmt
  Stmt -> foreach point in Area Stmt
  \end{lstlisting}

  Actions:
  \begin{lstlisting}
  Act -> visit Loc | pick Itm | drop Itm
  Act -> select Itm
  Act -> strict Act
  \end{lstlisting}

  Locations:
  \begin{lstlisting}
  Area -> Area containing Itm
  Area -> Area and Area | Area or Area
  Area -> world   // every point on the map
  Area -> Pnt | [Pnt, Pnt, ..., Pnt]
  Pnt -> [x, y]
  \end{lstlisting}

  Items:
  \begin{lstlisting}
  Itm -> item | item Fltr
  Fltr -> Fltr and Fltr | Fltr or Fltr | not Fltr
  Fltr -> has color C | has shape S
  C -> red | blue | green | yellow
  S -> triangle | square | circle
  \end{lstlisting}
  \caption{Syntax of core language (subset)}
  \label{fig:core-syntax}
\end{figure}
In addition, our language also allows to express sets of areas and
conditionals which we omit here for space reasons.

With this core langauage, we can express for example the following:
\begin{itemize}
	\item Drop an item to any field that contains both a red and a circle-shaped item
    (possibly the same item):
    \begin{lstlisting}
foreach point in {world containing item has color red} and
{world containing item has shape circle } {visit point; drop item}
	   \end{lstlisting}

  \item If possible, form a horizontal line on the floor out of all items robot
    currently has and starting at robot's current position:
    \begin{lstlisting}
strict {while robot has item {drop item; move right}};
strict {while robot has item {drop item; move left}}
    \end{lstlisting}

	\item Keep bringing circle-shaped items to room1 until there is a red item in room1:
    \todo[inline]{(Eva) I have thought to not include while in the syntax, as everything
    we include we need to explain which takes space. This example uses it though.}
    \begin{lstlisting}
while not { item has color red at room1}
  {visit {world containing item has shape circle} minus room1;
pick item has shape circle; visit room1; drop item has shape circle}
    \end{lstlisting}
\end{itemize}


\subsection{Semantics}

The model of our world consists of a tuple $(M, I, r)$ where
$M$ is a two-dimensional grid, $I$ is the set of items, and $r$ is the robot.
Each item $i \in I$ is has an associated color, shape and a unique identifier.
The robot has a position in $M$ and holds a set of items. All items are either held by
the robot, or can be mapped to a position in $M$ ($pos: i \in I \to M$).
The semantics of our core language is summarized in~\autoref{fig:core-semantics}.
\todo[inline]{(Eva) To be continued... Not sure yet how to present this.}
\begin{figure}
  \begin{align*}
  \sem{item Fltr}{\{i \in I: Filter(i) == true \}} \\
  \sem{has color c}{i.color == c}\\
  ...
  \end{align*}
  \caption{Semantics of the core language}
  \label{fig:core-semantics}
\end{figure}
