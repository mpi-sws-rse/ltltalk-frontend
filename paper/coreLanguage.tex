\section{Core Language and Planning}
\label{sec:coreLanguage}

\tool starts with a \emph{core language} for specifying temporal tasks for a robot in
the abstract world.
\autoref{fig:core-syntax} shows the most interesting subset of the syntax
of the core language.
% the full syntax is available at \url{syntaxurl}.
While the abstract world in which our robot operates may seem relatively simple,
the core language allows expressing many interesting scenarios.
For example, it allows us to express the tasks in~\autoref{fig:core-examples}.

\begin{figure}
  Control flow:
  \begin{lstlisting}
Stmt $\to$ Act $\mid$ Stmt; Stmt $\mid$ repeat Num times Stmt
      $\mid$ foreach point in Area Stmt $\mid$ if Cnd Stmt $\mid$ while Cnd Stmt
  \end{lstlisting}
  Actions:
  \begin{lstlisting}
Act     $\to$ visit Area $\mid$ visit Area while avoiding Area
         $\mid$ move right $\mid$ move left $\mid$ $\ldots$
         $\mid$ ItemAct QItm $\mid$ strict Act
ItemAct $\to$ pick $\mid$ drop
  \end{lstlisting}
  Locations:
  \begin{lstlisting}
Area $\to$ world $\mid$ Pnt $\mid$ [Pnt, Pnt, ..., Pnt] $\mid$ Area containing Itm
     $\mid$ Area and Area $\mid$ Area or Area $\mid$ Area minus Area
Pnt  $\to$ [Num, Num] $\mid$ point
  \end{lstlisting}
  Items:
  \begin{lstlisting}
QItm $\to$ every Itm $\mid$ Itm
Itm  $\to$ item $\mid$ item Fltr
Fltr $\to$ has Prop $\mid$
     $\mid$ Fltr and Fltr $\mid$ Fltr or Fltr $\mid$ not Fltr
Prop $\to$ color C $\mid$ shape S
C    $\to$ red $\mid$ blue $\mid$ green $\mid$ yellow $\mid$ $\ldots$
S    $\to$ triangle $\mid$ square $\mid$ circle $\mid$ $\ldots$
  \end{lstlisting}
  Conditions:
  \begin{lstlisting}
Cnd $\to$ Itm at Area $\mid$ robot has Item $\mid$ robot at Area $\mid$ possible Stmt
  \end{lstlisting}
  \caption{Syntax of core language (subset). Reserved constants and variable
  names are marked in italic.}
  \label{fig:core-syntax}
\end{figure}

\begin{figure}
\begin{example}\label{ex:example1}
Go to a red item and pick it up $\ldots$
\begin{lstlisting}
visit world containing item has color red;
pick item has color red
\end{lstlisting}
$\ldots$ but avoid all circular items along the way
\begin{lstlisting}
visit world containing item has color red
    while avoiding world containing item has shape circle;
pick item has color red
\end{lstlisting}
\vspace{-10pt}
\end{example}

\begin{example}\label{ex:example2}
Gather all red items
\begin{lstlisting}
foreach point in world containing item has color red
    { visit point; pick every item has color red }
\end{lstlisting}
\vspace{-10pt}
\end{example}

\begin{example}\label{ex:example3}
%\item
If possible, form a horizontal line on the floor out of all the items
  the robot currently has (starting at robot's current position and to the right)
    \begin{lstlisting}
strict {while robot has item {drop item; move right}};
    \end{lstlisting}
    \vspace{-10pt}
\end{example}

\begin{example}\label{ex:example4}
Place as many items in a horizontal line as fit.
  \begin{lstlisting}
drop item;
while possible {move right; drop item} {move right; drop item}
  \end{lstlisting}
  \vspace{-10pt}
  \end{example}
\caption{Example commands in the core language}
\label{fig:core-examples}
\end{figure}


Intuitively, the core language interprets a program as a
\emph{temporal goal} for the robot in a grid world.
% \footnote{A full formal definition of our semantics can be found at \url{semanticsurl}.}
The model of the grid world consists of a tuple $(M, I, r)$ where
$M$ is a two-dimensional grid of \emph{points} divided into free points and obstacles,
$I$ is a set of \emph{items}, and $r$ is the robot.
Each item $i \in I$ has associated attributes such as color, shape, and a unique identifier.
Additionally, if it is not held by the robot, it has a position which is a point in $M$.
The robot $r$ has a position in $M$ and holds a possibly empty set of items.

The core language has a set of actions that manipulate the world and combines
actions through standard imperative constructs such as sequencing, conditionals, or loops.
We focus here on the non-standard parts of the language.

%propositions
The propositions of the logical language specify either sets of items or sets of
points, and can be combined using Boolean operations as usual.
The syntactic class ``Locations'' describes sets of points in the grid $M$.
The user can, for instance, start with the entire grid $M$ using
the keyword \lstinline{world} and filter the grid points to those which
contain interesting items using the \lstinline{containing} clause
(see Example~\ref{ex:example1}).
Similarly, the syntactic class ``Items'' describes sets of items.
All items can be selected with the \lstinline{every} keyword and specific items can be chosen
by filtering based on (possibly Boolean combinations) of attributes
(Example~\ref{ex:example2}).

\emph{Actions} modify the state of the world. For example, \lstinline{pick} changes the
world from a state where there is an item $i$ in the current position of the
robot to a world in which the item is being carried by the robot.
Actions can be \emph{temporal}.
The temporal action \lstinline{visit T}, for a set $T\subseteq M$ of points,
requires that the robot is moved to some position in $T$.
The temporal action \lstinline{visit $T$ while avoiding $A$} additionally requires
that along the way the robot never visits any point in $A\subseteq M$
(see Example~\ref{ex:example1}).
They can be written in linear temporal logic (LTL) as $\Diamond T$ and $\until{\lnot A}{T}$,
respectively.
The language is expressive enough to subsume LTL with propositions ranging over points and
items.
% \todo[inline]{IG: how do we encode next (X) operator for visiting a location?}
% RM: move left or move right .. etc


In a complex command, only some part of a command may be realizable.
Consider Example~\ref{ex:example3}: the robot
may be able to move right once but not as many times as it has items.
The default behavior is lenient in that it executes those parts of the command
which are realizable and prints a warning about those which are not.
The \lstinline{strict} modifier allows to specify more rigid behavior that
either performs the \emph{complete} action or no part of it.

In a slight modification of the scenario above, if we want to place as many
items to the right as possible, as in Example~\ref{ex:example4}, then we can use
the \lstinline{while possible S T} construct which executes $T$ repeated while
$S$ is realizable.


% \subsection{Planner and User Feedback}

Given a goal expressed in the core language,
\tool generates an execution plan for the robot to satisfy it in the grid world.
Currently, \tool's planner is based on A* search~\cite{Astar} with Christofides algorithm~\cite{christofides}
for iterated reachability.
However, the planner interface is generic and one can use
a different planner \cite{golog,pddl,hadasLTLMop,ankushDrona,antlab}.
Fast planning is crucial for user interaction and we found
A* to be fast and sufficient to find plans.

The plan generated is shown to the user visually in \tool's graphical interface
by simulating the robot's moves step by step in the UI.
This visual feedback is important to verify that a possibly complex command given to
the robot works as expected.
Furthermore, ambiguities in the grammar can lead to several alternative plans
being generated; the UI enables the user to select the plan which best matches
his or her intentions.



