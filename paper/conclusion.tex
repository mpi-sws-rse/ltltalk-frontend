\section{Conclusion}

% Effectively communicating with robots has been a long-standing research problem.
We have shown that naturalizing a domain-specific programming language is well suited
to provide a natural language interface to robot task specifications.
\tool provides the precision, expressivity, and extensibility of a programming
language, while ensuring a natural experience for humans.
\tool adapts its language to its users by learning new concepts from them. 
The results of our initial evaluation are encouraging and suggest
that a formal language for instructing robots can be turned with community
effort into a \emph{domain specific natural language}.

To accomplish our goal to its full extent, a few challenges remain to be solved.
The first one is lowering the entry bar to the system in its early phase, i.e.
learning the core language.
This could be accomplished by adding syntactic sugar or a specialized interface,
while keeping the rich formalism as a latent representation of
the space of robot tasks.
% A second direction is to include more sophisticated NLP techniques on top of the
% naturalization process such as lemmatization, rephrasing of definition heads, and
% rephrasing of the initial formal language.
Finally, the user experience of \tool can be improved to offer explanations for
unparseable utterances, better depict alternate interpretations of an utterance
which have the same execution, or auto-complete for available pre-defined concepts.
